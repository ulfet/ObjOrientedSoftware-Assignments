\clearpage
\section{DDD vs. OO Domain Model}
		
		\subsection{Discuss the challenges you faced while identifying the domain concepts,
			components, classes, categories and also while designing the domain model.}
		
		It was pretty straightforward to identify the core domain and the two support domains, as the supporting domains where already distinguished from the basic concept by the text of the exercise.
		As for the concepts within a bounded domain, one had to \textit{free} himself from technical thinking before, as it is of no importance at this stage how the data structures and relationships can be implemented.
		As an example, Floorplan and Placement are not directly connected within the Core Domain, albeit a placement procedure would be highly connected to the Floorplan.
		As a second example, static elements like Wall, Door and Window are not connected to placement, albeit there would be nearly no conceptually difference between placing a Door or a Chair, besides a difference in constraints: A door must be placed within a wall, while a chair must not.
		A second challenge was not to interleave the concepts too much.
		Regarding identifying the actual concepts, it was not that difficult because of the Noun-Verb analysis beforehand.
		Also, the specification in the exercise of the software to be designed was pretty clear about that.
		\\ \ \\
		To design the corresponding classes, their attributes and relationships was not that easy, as a lot more details must be thought of that lead to different ways of designing the system.
		Also, all three domains had to be incorporated into it.
		The most difficult part of the OO Model was the Core Domain.
		We tried to identify different patterns to come to a reasonably robust model, for example Furniture is modeled as a Flyweight as the position is stored in the Floorplan and not the Furniture itself.
		An additional difficulty was distinguishing between InWallElements and Furniture.
		As for the Domain Model, they are clearly different things sharing nothing besides being present within a Floorplan.
		At first, we thought of letting both classes inherit from a common base class, but later opted against it, as it would lead to additional complexity computing constraints.

		\subsection{Discuss the advantages or disadvantages of using DDD and OO for modelling
			the domain.}

		The advantage of DDD in an early design stage is structuring the demands in a way that makes it possible to put each concept into context of the whole application.
		It models the application as it is seen from the perspective of and educated user, interacting with it.
		Is it is important to do this without getting too technical in an early stage, as this could lead to work being wasted as early OO Concepts could be \textit{built on sand}.
		In later design stages, DDD simplifies too much.
		Some concepts become classes, some are created by a combination of classes, sometimes multiple concepts even can be implemented by a single base class and derived classes with very little extensions.
		Developing an application too near to the DDD, without applying OO as it could be, leads to unmaintainable, duplicated code that does not take advantage of design patterns.
		To summarise: Both concepts should be applied at the design stages they belong to, both taking advantage from each other.