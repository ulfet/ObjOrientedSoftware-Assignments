\clearpage
\section{Use \noindent Case Modelling}
	
		\subsection{ Based on your domain model create scenarios by modeling the main use \noindent Cases
			and actors as one or more UML use \noindent Case diagram.}
		
			\noindent Case \#1:\\
			(floorplan import)
			importing a background image into the application to be a floorplan\\
			
			
			\noindent Case \#2:\\
			(floorplan export)
			exporting a plan with all the furnitures inside\\
			
			
			\noindent Case \#3:\\
			(catalogue extension)
			importing an image into the application to be a furniture\\
			
			
			\noindent Case \#4:\\
			(catalogue extension)
			drawing in the application to be a furniture\\
			
			
			\noindent Case \#5:\\
			(floorplan, provide information)
			the size of the flat, free vs occupied space, and in-depth info on that\\
			
			
			\noindent Case \#6:\\
			(floorplan validation)
			constraints and rules are applied to check if the arrangement is logically and physically sound\\
			
			
			\noindent Case \#7:\\
			(SSO Login Features)
			in order to allow users to login to our social platform, they are allowed to login via other respected accounts such as Gmail and Facebook\\
			SSO = single-sign-on\\
			
			\noindent Case \#8:\\
			(user, comment)
			users should be allowed to comment on the other published projects\\
			
			
			\noindent Case \#9:\\
			(content moderation)
			user-generated contents have to be inspected before they are allowed to publish (the uploader can see the upload on our website, but others would not until an admin approves the content)
			sub\noindent Cases:\\ floor related (copyright or inappropriate designs), comment-related (laws of speech, inappropriate context)\\
			
			
			\noindent Case \#10:\\
			(user, design)
			users should be allowed to rotate and move furnitures inside a floorplan\\
		
		
		
		\subsection{Use the tabular textual description template uploaded in the exercise materials
			to detail the use \noindent Cases you have identified.}
		
		\clearpage	
		
			Use Case \#1:
			
			\begin{table}[H]
				\scalebox{1.0}{
				\begin{tabularx}{\textwidth}{|l|X|}
					\hline
					Title & floorplan import from image file\\
					\hline
					Primary actor& user\\
					\hline
					Secondary& none\\
					 actor(s)& \\
					 
					\hline
					Goal& having the image plan in  the form that can be used in our application imported from an image\\
					\hline
					Short description& user should be able to use an image of hers as an input for a floorplan, which would be interpreted and recreated as a floorplan structure that can be used in floorplan and furniture arrangement\\
					\hline
					Trigger& click of the user on the button "Submit Image as Floorplan" displayed on the web interface\\
					\hline
					Preconditions&  
						\begin{tabular}[l]{@{}c@{}}
							1) user should browse and select a file from her computer.\\
							2) the file should be in compatible type (JPEG, PNG, BMP)
						\end{tabular}
						\\
					\hline
					Postconditions & None.\\
					\hline
					Main Flow& 
						\begin{tabular}[x]{@{}l@{}}
							1) User selects "Import" tool.\\
							2) Using "Browse" button, user finds the image file that would be imported.\\
							3) User clicks "Submit Image as Floorplan" button.\\
							4) The file is sent via PUSH (HTTP command) to the application server.\\
							5) Application server checks if the sent file is valid file.\\
							5a) if so, the application would convert it to a format that is possible\\
							 to process via the application itself.\\
							6) User is notified about the outcome of the process.
						\end{tabular}
					\\
					\hline
					Additional information & The sent image file size should not exceed 10 MB.\\
					\hline
					Deviation Flows & \\
					\hline
					Deviation Flow 1 & 
						\begin{tabular}[x]{@{}l@{}}
							(earlier steps same as before)\\
							5) Application server checks if the sent file is valid file.\\
							5b) file is either invalid or it is not possible to convert it to a floorplan.\\
							6) User is notified with the information on the image that it is\\
							 not valid.
						\end{tabular}\\
					
					\hline
					Additional Information& \\
					\hline
				\end{tabularx}
				}
			\end{table}
			
			% EXAMPLE USE CASE TABLE
			Use Case \#2:
			\begin{table}[H]
				\scalebox{1.0}{
				\begin{tabularx}{\textwidth}{|l|X|}
					\hline
					Title & floorplan export\\
					\hline
					Primary actor& user\\
					\hline
					Secondary& none\\
					actor(s)& \\
					
					\hline
					Goal& exporting a plan with all the furnitures inside in a format that can be later imported\\
					\hline
					Short description& A user can export one of his projects, or other peoples' projects into a format that can be imported to the application later.\\
					\hline
					Trigger& click of the user on the button "Export Floorplan" displayed on the web interface\\
					\hline
					Preconditions&  
					\begin{tabular}[l]{@{}c@{}}
						1) the project has to be the user's own project or a public project.\\
					\end{tabular}
					\\
					\hline
					Postconditions & None.\\
					\hline
					Main Flow& 
					\begin{tabular}[x]{@{}l@{}}
						1) User navigates to a project via her web browser.\\
						2) User clicks on the "Export Floorplan" button on the respective \\
						webpage\\
						3) The application runs in the server to convert the floorplan with all\\
						 the details into a text file, with necessary additional files if needed, and \\
						 zips them.\\
						4) The application sends the zip file to the user's web client.\\
					\end{tabular}
					\\
					\hline
					Additional information & \\
					\hline
					Deviation Flows & \\
					\hline
					Deviation Flow 1 & 
					\begin{tabular}[x]{@{}l@{}}
						1) User navigates to a project link in which she has no privilege to \\
						display.\\
						2) System does not allow displaying and exporting of the project, \\
						and inform the user about this issue.\\
						
					\end{tabular}\\
					
					\hline
					Additional Information& \\
					\hline
				\end{tabularx}
				}
			\end{table}
		
			\clearpage
		
			Use Case \#3:
			\begin{table}[H]
				\scalebox{1.0}{
				\begin{tabularx}{\textwidth}{|l|X|}
					\hline
					Title &furniture importing via image file\\
					\hline
					Primary actor& user\\
					\hline
					Secondary& none\\
					actor(s)& \\
					
					\hline
					Goal&allowing user to add her own images of furniture to the system\\
					\hline
					Short description& User can take a picture of the furniture she wants to use to arrange her floor, and this picture can be uploaded to the system to be used in our application.\\
					\hline
					Trigger& click of the user on the button "Submit Image as Furniture" displayed on the web interface\\
					\hline
					Preconditions&  
					\begin{tabular}[l]{@{}c@{}}
						None
					\end{tabular}
					\\
					\hline
					Postconditions & None.\\
					\hline
					Main Flow& 
					\begin{tabular}[x]{@{}l@{}}
						1) User navigates to a project of hers.\\
						2) On the project page, she chooses "Import Tools".\\
						3) On the Import Tools menu, she browses and selects one image she \\
						wants to upload.\\
						4) User clicks on the button "Submit Image as Furniture" displayed on \\
						the web interface\\
						5) The file is sent via PUSH (HTTP command) to the application server.\\
						6) Application server checks if the sent file is valid file.\\
						6a) if so, the system informs the user on the positive outcome.
					\end{tabular}
					\\
					\hline
					Additional information & The sent file size should not exceed 10 MB. \\
					\hline
					Deviation Flows & \\
					\hline
					Deviation Flow 1 & 
					\begin{tabular}[x]{@{}l@{}}
						(previous steps same as before)\\
						6b) if not valid, then the system would inform the user about the\\
						 negative outcome.\\
					\end{tabular}\\
					
					\hline
					Additional Information& \\
					\hline
				\end{tabularx}
				}
			\end{table}
		
			Use Case \#4:
			\begin{table}[H]
				\scalebox{1.0}{
				\begin{tabularx}{\textwidth}{|l|X|}
					\hline
					Title & Drawing a Furniture\\
					\hline
					Primary actor& user\\
					\hline
					Secondary& none\\
					actor(s)& \\
					
					\hline
					Goal& Allowing user to draw sketches that are to be converted into furnitures to be used in the application\\
					\hline
					Short description& Using the website of the application, a user can draw a furniture. After the drawing is finalized, user can opt to convert the drawing into a furniture she can use in the future.\\
					\hline
					Trigger& \\
					\hline
					Preconditions&  
					\begin{tabular}[l]{@{}c@{}}
						None
					\end{tabular}
					\\
					\hline
					Postconditions & None.\\
					\hline
					Main Flow& 
					\begin{tabular}[x]{@{}l@{}}
						1) User navigates to the "Drawing" page. \\
						2) User draws a furniture on her own imagination.\\
						3) User finalizes drawing.\\
						4) After saving, user is asked whether she wants to convert the\\
						 drawing into a furniture.\\
						5a) If positively answered, the drawing would be converted into a \\
						furniture, and would be added to the user's library.\\
						6) The user would be informed on the outcome of the conversion, and \\
						also about where the new furniture can be accessed.
					\end{tabular}
					\\
					\hline
					Additional information & \\
					\hline
					Deviation Flows & \\
					\hline
					Deviation Flow 1 & 
					\begin{tabular}[x]{@{}l@{}}
						(previous steps same as before)\\
						
						5b) if negatively answered, the drawing will be stored as it is, \\
						it won't be converted into a furniture.
						
					\end{tabular}\\
					
					\hline
					Additional Information& \\
					\hline
				\end{tabularx}
				}
			\end{table}
			
			\clearpage
		
			Use Case \#5:
			\begin{table}[H]
				\scalebox{1.0}{
				\begin{tabularx}{\textwidth}{|l|X|}
					\hline
					Title & provision of information about a floorplan\\
					\hline
					Primary actor& user\\
					\hline
					Secondary& none\\
					actor(s)& \\
					
					\hline
					Goal& to provide a user information about a flat\\
					\hline
					Short description& the size of the flat, free space, space occupied by furniture has to be provided to a user.\\
					\hline
					Trigger& A change in the layout of the floor\\
					\hline
					Preconditions&   
					\begin{tabular}[l]{@{}c@{}}
						None
					\end{tabular}
					\\
					\hline
					Postconditions & None.\\
					\hline
					Main Flow& 
					\begin{tabular}[x]{@{}l@{}}
						1) User navigates to the view of the floorplan.\\
						2) The said info about the floor is displayed on the GUI.\\
						3) Whenever a change is applied to the layout of the floor by user \\
						(moving of the furniture, addition or removal of the furniture, change in \\
						the floor properties, etc.), those information is recalculated on the\\
						 client-side.\\
						4) After the calculation, the GUI is updated to reflect the changes.\\
					\end{tabular}
					\\
					\hline
					Additional information & \\
					\hline
					Deviation Flows &  (there is none)\\
					\hline
				\end{tabularx}
				}
			\end{table}
		
			Use Case \#6:
			\begin{table}[H]
				\scalebox{1.0}{
				\begin{tabularx}{\textwidth}{|l|X|}
					\hline
					Title & Checking of the Constraints and Rules on a Floorplan\\
					\hline
					Primary actor& user\\
					\hline
					Secondary& none\\
					actor(s)& \\
					
					\hline
					Goal& to check whether the current state of a floorplan is logically and physically sound\\
					\hline
					Short description& Checking physical limits of the layout is needed to see whether such arrangement is possible.\\
					\hline
					Trigger& A change in the layout of the floor\\
					\hline
					Preconditions&  
					\begin{tabular}[l]{@{}c@{}}
						None.
					\end{tabular}
					\\
					\hline
					Postconditions & None.\\
					\hline
					Main Flow& 
					\begin{tabular}[x]{@{}l@{}}
						1) User navigates to the view of the floorplan.\\
						2) Whenever a change is applied to the layout of the floor by user \\
						(moving of the furniture, addition or removal of the furniture, change in \\
						the floor properties, etc.), the sanity check is performed on the server.\\
						3a) If the submitted layout is sound, client application is informed that \\
						the change is alright, and it is safe to continue.
					\end{tabular}
					\\
					\hline
					Additional information & \\
					\hline
					Deviation Flows & \\
					\hline
					Deviation Flow 1 & 
					\begin{tabular}[x]{@{}l@{}}
						(previous steps same as before)\\
						3b) If the submitted layout is not sound, client application is informed \\
						that the change is not alright.\\
						4) Client application reverts back the last change, and informs user \\
						that the action is not performable.
					\end{tabular}\\
					
					\hline
					Additional Information& \\
					\hline
				\end{tabularx}
				}
			\end{table}
		
			\clearpage
		
			Use Case \#7:
			\begin{table}[H]
				\scalebox{1.0}{
				\begin{tabularx}{\textwidth}{|l|X|}
					\hline
					Title &\\
					\hline
					Primary actor& user\\
					\hline
					Secondary& none\\
					actor(s)& \\
					
					\hline
					Goal&\\
					\hline
					Short description& \\
					\hline
					Trigger& \\
					\hline
					Preconditions&  
					\begin{tabular}[l]{@{}c@{}}
						1) \\
						2) 
					\end{tabular}
					\\
					\hline
					Postconditions & None.\\
					\hline
					Main Flow& 
					\begin{tabular}[x]{@{}l@{}}
						1) \\
						2) \\
						3) \\
						4) \\
						5) \\
						6) 
					\end{tabular}
					\\
					\hline
					Additional information & \\
					\hline
					Deviation Flows & \\
					\hline
					Deviation Flow 1 & 
					\begin{tabular}[x]{@{}l@{}}
						1)\\
						2)\\
						
					\end{tabular}\\
					
					\hline
					Additional Information& \\
					\hline
				\end{tabularx}
				}
			\end{table}
		
			Use Case \#8:
			\begin{table}[H]
				\scalebox{1.0}{
				\begin{tabularx}{\textwidth}{|l|X|}
					\hline
					Title &\\
					\hline
					Primary actor& user\\
					\hline
					Secondary& none\\
					actor(s)& \\
					
					\hline
					Goal&\\
					\hline
					Short description& \\
					\hline
					Trigger& \\
					\hline
					Preconditions&  
					\begin{tabular}[l]{@{}c@{}}
						1) \\
						2) 
					\end{tabular}
					\\
					\hline
					Postconditions & None.\\
					\hline
					Main Flow& 
					\begin{tabular}[x]{@{}l@{}}
						1) \\
						2) \\
						3) \\
						4) \\
						5) \\
						6) 
					\end{tabular}
					\\
					\hline
					Additional information & \\
					\hline
					Deviation Flows & \\
					\hline
					Deviation Flow 1 & 
					\begin{tabular}[x]{@{}l@{}}
						1)\\
						2)\\
						
					\end{tabular}\\
					
					\hline
					Additional Information& \\
					\hline
				\end{tabularx}
				}
			\end{table}
		
			\clearpage
		
			Use Case \#9:
			\begin{table}[H]
				\scalebox{1.0}{
				\begin{tabularx}{\textwidth}{|l|X|}
					\hline
					Title &\\
					\hline
					Primary actor& user\\
					\hline
					Secondary& none\\
					actor(s)& \\
					
					\hline
					Goal&\\
					\hline
					Short description& \\
					\hline
					Trigger& \\
					\hline
					Preconditions&  
					\begin{tabular}[l]{@{}c@{}}
						1) \\
						2) 
					\end{tabular}
					\\
					\hline
					Postconditions & None.\\
					\hline
					Main Flow& 
					\begin{tabular}[x]{@{}l@{}}
						1) \\
						2) \\
						3) \\
						4) \\
						5) \\
						6) 
					\end{tabular}
					\\
					\hline
					Additional information & \\
					\hline
					Deviation Flows & \\
					\hline
					Deviation Flow 1 & 
					\begin{tabular}[x]{@{}l@{}}
						1)\\
						2)\\
						
					\end{tabular}\\
					
					\hline
					Additional Information& \\
					\hline
				\end{tabularx}
				}
			\end{table}
		
			Use Case \#10:
			\begin{table}[H]
				\scalebox{1.0}{
				\begin{tabularx}{\textwidth}{|l|X|}
					\hline
					Title &\\
					\hline
					Primary actor& user\\
					\hline
					Secondary& none\\
					actor(s)& \\
					
					\hline
					Goal&\\
					\hline
					Short description& \\
					\hline
					Trigger& \\
					\hline
					Preconditions&  
					\begin{tabular}[l]{@{}c@{}}
						1) \\
						2) 
					\end{tabular}
					\\
					\hline
					Postconditions & None.\\
					\hline
					Main Flow& 
					\begin{tabular}[x]{@{}l@{}}
						1) \\
						2) \\
						3) \\
						4) \\
						5) \\
						6) 
					\end{tabular}
					\\
					\hline
					Additional information & \\
					\hline
					Deviation Flows & \\
					\hline
					Deviation Flow 1 & 
					\begin{tabular}[x]{@{}l@{}}
						1)\\
						2)\\
						
					\end{tabular}\\
					
					\hline
					Additional Information& \\
					\hline
				\end{tabularx}
				}
			\end{table}
		
		\clearpage
			
			
		\subsection{In the first step, you should have identified around 10 use Cases. Please specify
			five of these using UML activity diagrams. Do mind, that these should not be
			trivial!}
		
		\subsection{Explain the advantages of using activity diagrams.}
		
		Although use-case diagrams are helpful for discussing how a system broadly behaves, they are not enough to show the underlying logic. With activity diagrams, a observer can see which component is sending what kind of message, in which order, and how the said message got a reply.\\
		
		In other words, activity diagram contains more in-depth and technical information compared to use-case diagrams, which might help a keen eye on the understanding the concept in a more detailed fashion.