\section{Components}

After making clear that there is a timeframe for successful completion of the tasks, and that coming earlier (a.k.a rushing) would not grant a user any benefit, we updated our storyboard as follows:

\subsection{SketchImporter}
	\begin{description}
		\item[Decision] \hfill \\ Bad suggestion
		\item[Justification] \hfill \\ It is too specific and hence not reusable. It should be ImageImporter, which would give user a choice whether it is a furniture piece or a floor plan.
		\item[Interfaces] \hfill \\ -
		\item[Interactions with other components] \hfill \\ -
		\item[Included classes or entities] \hfill \\ -
		\item[Integration with JHotDraw] \hfill \\ -
	\end{description}

\subsection{FlatModeler}
	\begin{description}
		\item[Decision] \hfill \\ Bad suggestion
		\item[Justification] \hfill \\ It is very general and would basically contain the entire functionality, whereas components are supposed to be smaller and limited to a certain type of functionality (like WallModeler).
		\item[Interfaces] \hfill \\ -
		\item[Interactions with other components] \hfill \\ -
		\item[Included classes or entities] \hfill \\ -
		\item[Integration with JHotDraw] \hfill \\ -
	\end{description}

\subsection{ImageExporter}
	\begin{description}
		\item[Decision] \hfill \\ Good suggestion
		\item[Justification] \hfill \\ It is small functionality part in a very defined context. One knows it is used for exporting images.
		\item[Interfaces] \hfill \\ public URIChooser createExportChooser(Application a, @Nullable View v)
		\item[Interactions with other components] \hfill \\ ExportFileAction
		\item[Included classes or entities] \hfill \\ URIChooser, Application, View
		\item[Integration with JHotDraw] \hfill \\ A button on a toolbar.
	\end{description}

\subsection{WebPublisher}
	\begin{description}
		\item[Decision] \hfill \\ Good component
		\item[Justification] \hfill \\ Again, the usage of this component is clear and understandable from the name.
		\item[Interfaces] \hfill \\ publishPlan(image, author)
		\item[Interactions with other components] \hfill \\ -
		\item[Included classes or entities] \hfill \\ Some communication classes, also some conversion classes needed.
		\item[Integration with JHotDraw] \hfill \\ A button on a toolbar.
	\end{description}

\subsection{FurnitureFactory}
	\begin{description}
		\item[Decision] \hfill \\ Bad suggestion
		\item[Justification] \hfill \\ Dependant on the context and hence not reusable.
		\item[Interfaces] \hfill \\ -
		\item[Interactions with other components] \hfill \\ -
		\item[Included classes or entities] \hfill \\ -
		\item[Integration with JHotDraw] \hfill \\ -
	\end{description}

\subsection{ChairTableGrouping}
	\begin{description}
		\item[Decision] \hfill \\ Bad suggestion
		\item[Justification] \hfill \\ If we want to group any other types, it is not reusable as it depends on the chair and the table type.
		\item[Interfaces] \hfill \\ -
		\item[Interactions with other components] \hfill \\ -
		\item[Included classes or entities] \hfill \\ -
		\item[Integration with JHotDraw] \hfill \\ -
	\end{description}

\subsection{ConstraintValidation}
	\begin{description}
		\item[Decision] \hfill \\ Bad suggestion
		\item[Justification] \hfill \\ Very general as there are different constraints for various types of furniture - one can fill the space under the table with something, which is not the case with a solid object like the drawer. Each object group should have its own constraint validation. Additionally, one should run the constraint validation after attempted placement of the object to avoid computation overhead.
		\item[Interfaces] \hfill \\ -
		\item[Interactions with other components] \hfill \\ -
		\item[Included classes or entities] \hfill \\ -
		\item[Integration with JHotDraw] \hfill \\ -
	\end{description}

Decide for each of the following suggestions if they are components and justify your
answer. In the case of a good component candidate extend the description with
Interfaces
Interactions with other components
Included classes or entities
Integration with JHotDraw

% \subsection{SketchImporter}
% This is a good choice for a component. It is a small part of the program, which could also be used in different drawing programs. Expect of the dependency to JHotDraw it has no further dependencies and therefore is well suited for an own component.

% In general this component should receive an image file and import this to the given drawing but the interface looks a bit different because of the implementation with JHotDraw: The component is implemented as a \textit{Tool} and is an extension of \textit{ImageTool}. But only the \textit{creationFinished} method is overwritten. The \textit{ImageTool} already provides a functionality to import an image to the drawing. With the extension this image is set to the background and positioned to the (0,0) point. As the \textit{ImageTool}, this component also requires an \textit{ImageFigure} as constructor parameter and interact directly to the current view of this JHotDraw application.

% \subsection{ImageExporter}
% The ImageExporter is not a good choice for a component. The exporting feature is a general feature for a drawing application and therefore it is more clever to integrate this to the framework directly. So it could be extended by different applications for different needs, for example different file formats to export.

% In JHotDraw is an exporter feature already implemented. It could be provided through the \textit{ExportFileAction}. In addition to this to hot spots have to be implemented: In the \textit{ApplicationModel} the method \textit{createExportChooser} have to be overwritten to show all available formats in the export dialog and the method \textit{write} in the \textit{View} have to be changed to save the drawing in the selected format.
